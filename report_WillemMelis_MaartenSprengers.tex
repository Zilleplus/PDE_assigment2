% !TEX TS-program = pdflatex
% !TEX encoding = UTF-8 Unicode

% This is a simple template for a LaTeX document using the "article" class.
% See "book", "report", "letter" for other types of document.

\documentclass[11pt]{article} % use larger type; default would be 10pt

\usepackage[utf8]{inputenc} % set input encoding (not needed with XeLaTeX)

%%% Examples of Article customizations
% These packages are optional, depending whether you want the features they provide.
% See the LaTeX Companion or other references for full information.

%%% PAGE DIMENSIONS
\usepackage{geometry} % to change the page dimensions
\geometry{a4paper} % or letterpaper (US) or a5paper or....
\geometry{margin=0.8in} % for example, change the margins to 2 inches all round
% \geometry{landscape} % set up the page for landscape
%   read geometry.pdf for detailed page layout information

\usepackage{graphicx} % support the \includegraphics command and options
\usepackage{caption}

% \usepackage[parfill]{parskip} % Activate to begin paragraphs with an empty line rather than an indent

%%% PACKAGES
\usepackage{booktabs} % for much better looking tables
\usepackage{array} % for better arrays (eg matrices) in maths
\usepackage{paralist} % very flexible & customisable lists (eg. enumerate/itemize, etc.)
\usepackage{verbatim} % adds environment for commenting out blocks of text & for better verbatim
\usepackage{subfig} % make it possible to include more than one captioned figure/table in a single float
% These packages are all incorporated in the memoir class to one degree or another...
\usepackage{amsmath}
% to get figures on the right spot (capital H parameter)
\usepackage{float}
% makes sure the paragraphs are seperated with an enter space
\usepackage[parfill]{parskip}

\usepackage{subfig}


%%% HEADERS & FOOTERS
\usepackage{fancyhdr} % This should be set AFTER setting up the page geometry
\pagestyle{fancy} % options: empty , plain , fancy
\renewcommand{\headrulewidth}{0pt} % customise the layout...
\lhead{}\chead{}\rhead{}
\lfoot{}\cfoot{\thepage}\rfoot{}

%%% SECTION TITLE APPEARANCE
\usepackage{sectsty}
\allsectionsfont{\sffamily\mdseries\upshape} % (See the fntguide.pdf for font help)
% (This matches ConTeXt defaults)

%%% ToC (table of contents) APPEARANCE
\usepackage[nottoc,notlof,notlot]{tocbibind} % Put the bibliography in the ToC
\usepackage[titles,subfigure]{tocloft} % Alter the style of the Table of Contents
\renewcommand{\cftsecfont}{\rmfamily\mdseries\upshape}
\renewcommand{\cftsecpagefont}{\rmfamily\mdseries\upshape} % No bold!

%used for matlab code
\usepackage{listings}
\usepackage{color} %red, green, blue, yellow, cyan, magenta, black, white
\definecolor{mygreen}{RGB}{28,172,0} % color values Red, Green, Blue
\definecolor{mylilas}{RGB}{170,55,241}

\usepackage[dutch]{babel}

\graphicspath{ {./figures/} }

%%% END Article customizations

%%% The "real" document content comes below...

\title{Report numerical simulation of differential equations - Homework assignment 2}

\author{Willem Melis, Maarten Sprengers}
%\date{} % Activate to display a given date or no date (if empty),
         % otherwise the current date is printed
\lstset{language=Matlab,%
	%basicstyle=\color{red},
	breaklines=true,%
	morekeywords={matlab2tikz},
	keywordstyle=\color{blue},%
	morekeywords=[2]{1}, keywordstyle=[2]{\color{black}},
	identifierstyle=\color{black},%
	stringstyle=\color{mylilas},
	commentstyle=\color{mygreen},%
	showstringspaces=false,%without this there will be a symbol in the places where there is a space
	numbers=left,%
	numberstyle={\tiny \color{black}},% size of the numbers
	numbersep=9pt, % this defines how far the numbers are from the text
	emph=[1]{for,end,break},emphstyle=[1]\color{red}, %some words to emphasise
	%emph=[2]{word1,word2}, emphstyle=[2]{style},    
}
\begin{document}
\maketitle
\newpage
\tableofcontents
\newpage

\section{Algemene structuur Matlab implementatie}
TODO, Dit doe ik zelf wel voor de code die ik schreef, normaal gezien tegen morgen avond zal dit opgevuld zijn.
\section{Integratie verschillende vergelijkingen}

	\subsection{warmte vergelijking}
		\begin{eqnarray}
			\frac{\partial u}{\partial t} = 
			\frac{\partial ^2 u}{\partial x^2} + 
			\frac{\partial ^2 u}{\partial  y^2} 	
		\end{eqnarray}
		\begin{eqnarray}
			\frac{U_{r,s}^{n+1} - U_{r,s}^n}{\Delta t} =
			\frac{U_{r-1,s}^n - 2 U_{r,s}^n + U_{r+1,s}}{\Delta x^2} + 
			\frac{U_{r,s-1}^n - 2 U_{r,s}^n + U_{,s+1} }{\Delta y^2} \\
			U_{r,s}^{n+1} =( 1-4\mu )U_{r,s}^n + \mu (U_{r-1,s}^n  + U_{r+1,s} + 
			U_{r,s-1}^n  + U_{r,s+1})
		\end{eqnarray}

\begin{lstlisting}[caption=Code Explicit Euler,label={lst:expl_euler}]
function [ unew ] = functin_integrate_heat( u,mu,h,j_int)
	unew = (1-4*mu)*u(j_int) + ... 
	     mu*(u(j_int+h) + u(j_int-h) +  u(j_int+1) + u(j_int-1));
	end
\end{lstlisting}
		In les 5 over het deel van partiële differentiaal vergelijking werd de stabiliteit voor dit schema bewezen. Dit bleek $\mu_x + \mu_y \leq \frac{1}{2}$ te zijn. Aangezien $\mu_x = \mu_y$ kunnen kan worden gesteld dat: $\mu_x=\mu_y=\frac{\Delta y}{\Delta x^2} \leq \frac{1}{2}$.
	\subsection{Golf vergelijking}
		\begin{eqnarray}
		\frac{\partial^2 u}{\partial t^2} = 
		\frac{\partial ^2 u}{\partial x^2} + 
		\frac{\partial ^2 u}{\partial  y^2} 	
		\end{eqnarray}

		\begin{eqnarray}
			\frac{U_{r,s}^{n-1} - 2U_{r,s}^{n}+U_{r,s}^{n+1} }{\Delta t^2} 
			=
			\frac{U_{r-1,s}^n - 2 U_{r,s}^n + U_{r+1,s}}{\Delta x^2} + 
			\frac{U_{r,s-1}^n - 2 U_{r,s}^n + U_{,s+1} }{\Delta y^2} \\
			U_{r,s}^{n+1} = ( 1-2\mu )U_{r,s}^n + \mu (U_{r-1,s}^n  + U_{r+1,s}
			U_{r,s-1}^n  + U_{r,s+1}) - U_{r,s}^{n-1}
		\end{eqnarray}
\begin{lstlisting}[caption=Code Explicit Euler,label={lst:expl_euler}]
function [ unew,uold ] = functin_integrate_wave( u,uprevious,mu,h,j_int)
	uold=u; % keep the old one, as its needed in the nex iteration
	unew = (1-2*mu)*u(j_int) + uprevious(j_int) + ...
		mu*( u(j_int+h) + u(j_int-h) + ...
		u(j_int+1) + u(j_int-1) );
end
\end{lstlisting}
	\subsection{Transport vergelijking}
		\begin{equation}
			\frac{\partial u}{\partial t} = \frac{\partial u}{\partial x} + \frac{\partial u}{\partial y}
		\end{equation}
		
		\begin{eqnarray}
			\frac{U_{r,s}^{n+1} - U_{r,s}^n}{\Delta t} = 
			\frac{U^n_{r-1,s} -U^n_{r,s}}{\Delta x} +
			\frac{U^n_{r,s-1} -U^n_{r,s}}{\Delta y} \\
			U_{r,s}^{n+1} = (1- 2 \mu)U^n_{r,s} + \mu (U^n_{r-1,s}+U^n_{r,s-1})
		\end{eqnarray}
\begin{lstlisting}[caption=Code Explicit Euler,label={lst:expl_euler}]
function [ unew ] = functin_integrate_transport( u,mu,h,j_int)
	unew = (1-2*mu)*u(j_int) + ... 
	mu*( u(j_int-h) + u(j_int-1)) ;
end
\end{lstlisting}
\clearpage
\section{Stabiliteit}
	\subsection{Warmte vergelijking}
	Zoals eerder vermeld is het stabiliteitsgebied van de warmtevergelijking  $\mu_x=\mu_y=\frac{\Delta y}{\Delta x^2} \leq \frac{1}{2}$. In de Matlab implementatie gebruiken we de constanten k en h, om dt en $\Delta x$ of $\Delta y$ te berekenen. ($\Delta t=\frac{tf}{k}$ en $(\Delta x)^2 = \frac{1}{h^2}$)
	\begin{equation}
		k_{grens} = \frac{t_f h^2}{\mu_grens}
	\end{equation}
	\begin{figure}[H]
		\centering
		\subfloat[h=20 en k=160 de grens van stabiliteit]{
			\includegraphics[width=0.47\textwidth]{part2_stable_heat.png}
			%\label{fig:}
		}
		\hfill
		\subfloat[h=20 en k=150 dit is een onstabiele keuze]{
			\includegraphics[width=0.47\textwidth]{part2_unstable_heat.png}
			%\label{fig:}
		}
	\caption{Demonstratie stabiliteit van warmte vergelijking en keuze h/k}
	\end{figure}
	
	\subsection{Golf vergelijking}
	De grens van $\mu$ moet nog bepaald worden warschnk is dit 0.5.
	\begin{figure}[H]
		\centering
		\subfloat[h=20 en k=40 de grens van stabiliteit]{
			\includegraphics[width=0.47\textwidth]{part2_stable_heat.png}
			%\label{fig:}
		}
		\hfill
		\subfloat[h=20 en k=25 de grens van stabiliteit]{
			\includegraphics[width=0.47\textwidth]{part2_unstable_heat.png}
			%\label{fig:}
		}
		\caption{Demonstratie stabiliteit van golf vergelijking en keuze h/k}
	\end{figure}
	\subsection{Transport vergelijking}
	Waarschijnlijk is de grens van stabiliteit van dit ding $\mu<0.5$ , dit heb ik experimenteel vastgesteld dus hier moet nog een bewijs .
	\begin{figure}[H]
		\centering
		\subfloat[h=20 en k=80 de grens van stabiliteit]{
			\includegraphics[width=0.47\textwidth]{part2_stable_transport.png}
			%\label{fig:}
		}
		\hfill
		\subfloat[h=20 en k=65 de grens van stabiliteit]{
			\includegraphics[width=0.47\textwidth]{part2_unstable_transport.png}
			%\label{fig:}
		}
		\caption{Demonstratie stabiliteit van transport vergelijking en keuze h/k}
	\end{figure}
\clearpage
\section{Nauwkeurigheid TODO}
Hier kan hetzelfde script van part2 gebruikt worden!, deze geeft namelijk een u terug voor de simulaties... gewoon nog de inf norm bepalen.. Alsook de analytische oplossingen om ze mee te vergelijken.
	\subsection{Warmte vergelijking}
	\subsection{Golf vergelijking}
	\subsection{Transport vergelijking}
\clearpage
\section{Oplossing op t=0.1 en t=0.2 TODO}
	\subsection{Warmte vergelijking}
	TODO bepaal h en k
	\subsection{Golf vergelijking}
	TODO bepaal h en k
	\subsection{Transport vergelijking}
	TODO bepaal h en k
	\subsection{Resultaat}
	los een bug op met de beginvoorwaarden, die gaan om 1 of andere reden altijd naar nul , bekijk deze lijn eens:u=f\_u\_0(jx,jy);!!
	
	Die functie super klein, dit is waarschijnlijk niet juist
	\begin{figure}[H]
		\centering
		\subfloat[h=? en k=?]{
			\includegraphics[width=0.47\textwidth]{part4_heat.png}
			%\label{fig:}
		}
		\hfill
		\subfloat[h=? en k=?]{
			\includegraphics[width=0.47\textwidth]{part4_wave.png}
			%\label{fig:}
		}
	
		\subfloat[h=? en k=?]{
			\includegraphics[width=0.47\textwidth]{part4_transport.png}
			%\label{fig:}
		}
		\caption{Resultaat integratie op t=0.1 en t=0.2, de linker plot is altijd t=0.1 en de rechter is t=0.2}
	\end{figure}
\end{document}
